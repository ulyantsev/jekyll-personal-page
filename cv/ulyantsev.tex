
\documentclass[11pt,a4paper,sans]{moderncv}        % possible options include font size ('10pt', '11pt' and '12pt'), paper size ('a4paper', 'letterpaper', 'a5paper', 'legalpaper', 'executivepaper' and 'landscape') and font family ('sans' and 'roman')

% moderncv themes
\moderncvstyle{banking}                            % style options are 'casual' (default), 'classic', 'banking', 'oldstyle' and 'fancy'
\moderncvcolor{red}                               % color options 'black', 'blue' (default), 'burgundy', 'green', 'grey', 'orange', 'purple' and 'red'
%\renewcommand{\familydefault}{\sfdefault}         % to set the default font; use '\sfdefault' for the default sans serif font, '\rmdefault' for the default roman one, or any tex font name
%\nopagenumbers{}                                  % uncomment to suppress automatic page numbering for CVs longer than one page

\usepackage{fontawesome5}

\usepackage{comment} 

% character encoding
\usepackage[utf8]{inputenc}                       % if you are not using xelatex ou lualatex, replace by the encoding you are using

% adjust the page margins
\usepackage[scale=0.75]{geometry}

%\setlength{\hintscolumnwidth}{3cm}                % if you want to change the width of the column with the dates
%\setlength{\makecvheadnamewidth}{10cm}            % for the 'classic' style, if you want to force the width allocated to your name and avoid line breaks. be careful though, the length is normally calculated to avoid any overlap with your personal info; use this at your own typographical risks...

% personal data
\name{Vladimir}{Ulyantsev}
%\title{Resumé title}                               % optional, remove / comment the line if not wanted
%\address{street and number}{postcode city}{country}% optional, remove / comment the line if not wanted; the "postcode city" and "country" arguments can be omitted or provided empty
\email{vl.ulyantsev@gmail.com}
\homepage{ulyantsev.com}
\social[linkedin]{vladimir-ulyantsev}
\social[github]{ulyantsev}
\social[googlescholar][scholar.google.ru/citations?user=uzE__rYAAAAJ]{Vladimir Ulyantsev}

%\extrainfo{Hi there! }

%\social[twitter]{jdoe}                             % optional, remove / comment the line if not wanted
%\extrainfo{additional information}                 % optional, remove / comment the line if not wanted
%\photo[64pt][0.4pt]{picture}                       % optional, remove / comment the line if not wanted; '64pt' is the height the picture must be resized to, 0.4pt is the thickness of the frame around it (put it to 0pt for no frame) and 'picture' is the name of the picture file
%\quote{Some quote}                                 % optional, remove / comment the line if not wanted

\newcommand{\itmo}{ITMO University}
\newcommand{\doi}[1]{\href{http://doi.org/#1}{#1}}

\begin{document}
\makecvtitle

I am a PhD in computer science, associate professor and a head of Computer Technologies Laboratory (ITMO University),
which now consists of 7 groups \& 130 members.
I have managed and participated in more than 10 R\&D projects, 
%including an industry-funded R&D project on formal methods for a major international telecommunications company,
co-authored more than 45 research papers, 
supervised more than 25 students (including ICPC champions).
%My main research interests are computer science, artificial intelligence, generative design and bioinformatics.
\smallskip

\cvitem{Research interests}{discrete optimization, machine learning, generative design, bioinformatics}
\cvitem{Selected skills}{research unit \& project management, faculty work, funds attraction}
\cvitem{Personal values}{independence, ambition, logic, responsibility, social well-being}
\cvitem{Hobbies}{swimming, skiing, surfing, singing, texas holdem}


\section{Career overview}
Born in 1990 in St.~Petersburg, Russia. I have been keen on math and programming since early childhood.
Started studying computer science and programming in ITMO University in 2007, also started my career with
developing and organizing programming contests. 
I've developed more than 100 problems, coaching teams for ICPC championships in Russia, USA, Uzbekistan.
\smallskip

In 2010 started a research career on developing evolutionary and 
Boolean satisfiability (SAT) based methods for solving NP-hard problems.
Defended a PhD thesis on automata synthesis using SAT and Constraint Satifaction Problem (CSP) solvers in 2015.
In 2014 also started working in the area of bioinformatics, developing tools 
for metagenomic WGS analysis in collaboration with Research and Clinical Center of Physical-Chemical Medicine, Moscow.
\smallskip

In parallel with a research career, established as a leader and supervisor.
Since 2012 participated in grant fundraising and project supervision.
After defending a PhD thesis, in 2016 was appointed as the head of a joint ITMO-JetBrains Research lab (12~members).
Since 2017 became the head of the Computer Technologies Lab, facilitated the 
expansion of the Lab from 10 members to 7 groups of a total of 130 members.
%\smallskip

%More detailed bio available here: \url{https://ulyantsev.com/bio/}
%\smallskip

\section{Work}

\cventry{Jul 2017--now}{lab head}{\itmo, Computer Technologies Laboratory}{}{}
        {decision making, strategic planning, resource management, fundraising, collaborations establishment}
\cventry{Sep 2016--now}{associate professor}{\itmo, Information Technologies and Programming Faculty}{}{}
        {supervising bachelor, master \& PhD students, leading projects on computer science, machine learning \& bioinformatics}
%\cventry{Jan 2020--now}{deputy dean for science}{\itmo, Information Technologies and Programming Faculty}{}{}{}
%\cventry{Nov 2021--Apr 2022}{CEO}{\itmo, Center for AI research}{}{}{}
\smallskip
\cventry{Jul 2016--Feb 2022}{head of optimization problems in software engineering group}{JetBrains Research}{}{}{}
\cventry{Sep 2013--Jul 2019}{head of discrete optimization group}{\itmo, Computer Technologies Laboratory}{}{}{}
\cventry{Apr 2010--Dec 2015}{researcher}{\itmo, Computer Technologies Laboratory}{}{}{}
\cventry{Mar 2008--Mar 2010}{programming contests development \& organization}{\itmo, Information Technologies and Programming Faculty}{}{}{}

\section{Education}
\cventry{2013--2015}{PhD in Computer Science}{\itmo }{Saint Petersburg, Russia}
        {}{Information Technologies and Programming Faculty, Computer Technologies Department\\
           Topic: Finite-state machine synthesis using SAT and CSP solvers} 
\cventry{2011--2013}{Master's degree in Applied Mathematics and Informatics}{\itmo }{Saint Petersburg, Russia}
        {}{Information Technologies and Programming Faculty, Computer Technologies Department}
\cventry{2007--2011}{Bachelor's degree in Applied Mathematics and Informatics}{\itmo }{Saint Petersburg, Russia}
        {}{Information Technologies and Programming Faculty, Computer Technologies Department}

\section{Software}

\cventry{2014--now}{developer, algorithms designer, team leader}{MetaFast}{\url{https://github.com/ctlab/metafast}}
        {}{METAgenome FAST analysis toolkit for feature extraction and calculating a number of statistics of metagenome sequences}
\cventry{2016--now}{team leader, algorithms designer}{MetaCherchant}{\url{https://github.com/ctlab/metacherchant}}
        {}{tool for analysing genomic environment of a nucleotide sequence within a metagenome}
\cventry{2018--now}{team leader, algorithms designer}{RECAST}{\url{https://github.com/ctlab/RECAST}}
        {}{pipeline for analysing metagenome time series and distinguish which reads of one metagenome sample are found in other samples}
\cventry{2017--now}{supervisor for optimization algorithms}{GADMA}{\url{https://github.com/ctlab/GADMA}}
        {}{tool for automatic inference of the joint demographic history of multiple populations from the genetic data}
\cventry{2010--2019}{developer, algorithms designer, team leader}{EFSM-tools}{\url{https://github.com/ctlab/EFSM-tools}}
        {}{toolset for finite-state machine (FSM) synthesis}
% can be added https://github.com/ctlab/DFA-Inductor-py

\section{Selected funded projects}

% AI Center
% mnl
% Huawei project?

\cventry{2021--2023}{principal investigator}{RSCF grant}{\url{https://rscf.ru/en/project/21-71-00051/}}
        {}{Feature extraction for de Bruijn graphs of groups of metagenomic samples 
           and whole-genome metagenome datasets classification using machine learning}

\cventry{2018--2020}{principal investigator}{RSCF grant}{\url{https://rscf.ru/en/project/18-71-00150/}}
        {}{Evolution strategies for hard SAT-instances decomposition 
           with application to cryptographic functions inversion}

\cventry{2018--2020}{principal investigator}{RFBR A-grant}{\texttt{18-07-01285}}
        {}{Machine learning methods for synthesizing finite-state models of control systems 
           with regard of temporal properties and timestamps based on propositional encoding}

\cventry{2018--2020}{senior researcher}{Joint ITMO-Aalto project}{\texttt{14.587.21.0032}}
        {}{Development of methods, tools and technologies for design, 
           verification and testing of reliable cyber-physical systems}

\cventry{2014--2015}{PhD student}{RFBR PhD grant}{\texttt{14-07-31337}}
        {}{Automated reliable software synthesis from test examples and temporal constraints 
           using automata-based programming}


% do I realy need this section?
\section{Notable supervised students}

\cvitem{2021}{Artem Ivanov (MS)}
\cvitem{2020}{Ilya Zakirzyanov (PhD), Artem Pavlenko (MS), Darya Zvyagintseva (BS)}
\cvitem{2019}{Artem Ivanov (BS), Roman Melnikov (BS)}
\cvitem{2018}{Ekaterina Noskova (MS), Artem Pavleko (BS)}
\cvitem{2017}{Ilya Zakirzyanov (MS), Vyacheslav Moklev (BS), Ilya Kachalsky (BS)}
\cvitem{2016}{Artem Vasilyev (MS)}
\cvitem{2015}{Ilya Zakirzyanov (BS), Mikhail Melnik (BS)}


\section{Selected bioinformatics papers}

\begin{enumerate}
%  \item\emph{Ivanova V, Chernevskaya E, Vasiluev P, Ivanov A, Tolstoganov I, Shafranskaya D, Ulyantsev V, Korobeynikov A, Razin S, Beloborodova N, Ulianov S, Tyakht A.}
%       Hi-C metagenomics in the ICU: exploring clinically relevant features of gut microbiome in chronically critically ill patients.
%       \textbf{Frontiers in Microbiology}, 2022.
%       \doi{10.3389/fmicb.2021.770323}
  \item\emph{Olekhnovich E, Ivanov A, Ulyantsev V, Ilina E.}
       Separation of donor and recipient microbial diversity allows determination of taxonomic and functional features of gut microbiota Restructuring following Fecal Transplantation.
       \textbf{mSystems}, 2021.
       \doi{10.1128/mSystems.00811-21}
  \item\emph{Noskova E, Ulyantsev V, Koepfli K-P, O’Brien S, Dobrynin P.}
       GADMA: Genetic Algorithm for inferring Demographic history of Multiple populations from Allele frequency spectrum data.
       \textbf{GigaScience}, 2020.
       \doi{10.1093/gigascience/giaa005}
  \item\emph{Zhernakova D et al.}
       Genome-wide sequence analyses of ethnic populations across Russia.
       \textbf{Genomics}, 2020.
       \doi{10.1016/j.ygeno.2019.03.007}
  \item\emph{Olekhnovich E, Vasilyev A, Ulyantsev V, Kostryukova E, Tyakht A.}
       MetaCherchant: analyzing genomic context of antibiotic resistance genes in gut microbiota.
       \textbf{Bioinformatics}, 2018.
       \doi{10.1093/bioinformatics/btx681}
  \item\emph{Dubinkina V, Ischenko D, Ulyantsev V, Tyakht A, Alexeev D.}
       Assessment of k-mer spectrum applicability for metagenomic dissimilarity analysis
       \textbf{BMC bioinformatics}, 2016.
       \doi{10.1186/s12859-015-0875-7}
  \item\emph{Ulyantsev V, Kazakov S, Dubinkina V, Tyakht A, Alexeev D.}
       MetaFast: fast reference-free graph-based comparison of shotgun metagenomic data.
       \textbf{Bioinformatics}, 2016.
       \doi{10.1093/bioinformatics/btw312}
\end{enumerate}

\section{Selected computer science papers}
\begin{enumerate}
  \item\emph{Zvyagintseva D, Sigurdsson H, Kozin V, Iorsh I, Shelykh I, Ulyantsev V, Kyriienko O.}
       Machine learning of phase transitions in nonlinear polariton lattices.
       \textbf{Communications Physics}, 2022.
       \doi{10.1038/s42005-021-00755-5}
  \item\emph{Semenov A, Chivilikhin D, Pavlenko A, Otpuschennikov I, Ulyantsev V, Ignatiev A.}
       Evaluating the hardness of SAT instances using evolutionary optimization algorithms.
       \textbf{CP 2021}.
       \doi{10.4230/LIPIcs.CP.2021.47}
  \item\emph{Chivilikhin D, Zakirzyanov I, Ulyantsev V.}
       BeBoSy: Behavior Examples meet Bounded Synthesis.
       \textbf{IEEE Access}, 2021.
       \doi{10.1109/ACCESS.2021.3057823}
  \item\emph{Zakirzyanov I, Morgado A, Ignatiev A, Ulyantsev V, Marques-Silva J.}
       Efficient Symmetry Breaking for SAT-Based Minimum DFA Inference.
       \textbf{LATA 2019}.
       \doi{10.1007/978-3-030-13435-8\_12}
  \item\emph{Pavlenko A, Buzdalov M, Ulyantsev V.}
       Fitness comparison by statistical testing in construction of SAT-based guess-and-determine cryptographic attacks.
       \textbf{GECCO 2019}.
       \doi{10.1145/3321707.3321847}
  \item\emph{Chivilikhin D, Ulyantsev V, Shalyto A, Vyatkin V.}
       Function block finite-state model identification using SAT and CSP solvers.
       \textbf{IEEE Transactions on Industrial Informatics}, 2019.
       \doi{10.1109/TII.2019.2891614}
  \item\emph{Ulyantsev V, Buzhinsky I, Shalyto A.}
       Exact finite-state machine identification from scenarios and temporal properties.
       \textbf{International Journal on Software Tools for Technology Transfer}, 2018.
       \doi{10.1007/s10009-016-0442-1}
  \item\emph{Zakirzyanov I, Shalyto A, Ulyantsev V.}
       Finding all minimum-size DFA consistent with given examples: SAT-based approach.
       \textbf{SEFM 2017}.
       \doi{10.1007/978-3-319-74781-1\_9}
  \item\emph{Ulyantsev V, Zakirzyanov I, Shalyto A.}
       BFS-based symmetry breaking predicates for DFA identification.
       \textbf{LATA 2015}.
       \doi{10.1007/978-3-319-15579-1\_48}
\end{enumerate}


\section{Achievements \& activities}

\begin{itemize}
  \item Programme committee member of AAAI'2021, AINL '2016 '2017 and SNR'2017  
  \item In 2020 together with Artem Ivanov won both tracks of PMI Metagenomics Diagnosis for Inflammatory Bowel Disease Challenge (MEDIC) 
        (\url{https://www.pmiscience.com/whats-new/winners-of-sbv-improver-machine-learning-diagnostic-challenge-announced})
  \item Delivered scientific and popular science lectures on various open stages, a small collection available here:
        \url{https://ulyantsev.com/talks/}
  \item In 2019 took part in a standup show, joking on scientific mindset
  \item Starting 2018 actively serving in ITMO University' councils, doing my best to evolve internal university policies
  \item In 2009 have been coaching ICPC team of A.F. Mozhaysky's Military-Space Academy, 
        that was the only time their team managed to get to semifinal
\end{itemize}


%\section{References}

% Stephen J O'Brian

%\section{Master thesis}
%\cvitem{title}{\emph{Title}}
%\cvitem{supervisors}{Supervisors}
%\cvitem{description}{Short thesis abstract}



%\section{Computer skills}
%\cvdoubleitem{category 1}{XXX, YYY, ZZZ}{category 4}{XXX, YYY, ZZZ}
%\cvdoubleitem{category 2}{XXX, YYY, ZZZ}{category 5}{XXX, YYY, ZZZ}
%\cvdoubleitem{category 3}{XXX, YYY, ZZZ}{category 6}{XXX, YYY, ZZZ}

%\section{Interests}
%\cvitem{hobby 1}{Description}
%\cvitem{hobby 2}{Description}
%\cvitem{hobby 3}{Description}


%\section{References}
%\begin{cvcolumns}
%  \cvcolumn{Category 1}{\begin{itemize}\item Person 1\item Person 2\item Person 3\end{itemize}}
%  \cvcolumn{Category 2}{Amongst others:\begin{itemize}\item Person 1, and\item Person 2\end{itemize}(more upon request)}
%  \cvcolumn[0.5]{All the rest \& some more}{\textit{That} person, and \textbf{those} also (all available upon request).}
%\end{cvcolumns}



% Publications from a BibTeX file using the multibib package

%\nocitebook{book1,book2}
%\bibliographystylebook{plain}
%\bibliographybook{publications}                   % 'publications' is the name of a BibTeX file
%\nocitemisc{misc1,misc2,misc3}
%\bibliographystylemisc{plain}
%\bibliographymisc{publications}                   % 'publications' is the name of a BibTeX file

% \cventry{year--year}{Job title}{Employer}{City}{}{Description line 1\newline{}Description line 2}

%\cvitem{Nov 2021--now}{\itmo, Center for AI research, CEO}
%\cvitem{Jan 2020--now}{\itmo, Information Technologies and Programming Faculty, Deputy Dean for Science}
%\cvitem{Jul 2017--now}{\itmo, Computer Technologies Laboratory, Head}
%\cvitem{Jul 2016--Feb 2022}{JetBrains Research, Head of Optimization Problems in Software Engineering Group}
%\cvitem{Sep 2013--Jul 2019}{\itmo, Computer Technologies Laboratory, Head of Formal Methods Group}
%\cvitem{Apr 2010--Dec 2015}{\itmo, Computer Technologies Laboratory, Researcher}
%\cvitem{Mar 2008--Mar 2010}{\itmo, Laboratory Assistant}%{some text}



\begin{comment}
\clearpage
%-----       letter       ---------------------------------------------------------
% recipient data
\recipient{Company Recruitment team}{Company, Inc.\\123 somestreet\\some city}
\date{January 01, 1984}
\opening{Dear Sir or Madam,}
\closing{Yours faithfully,}
\enclosure[Attached]{curriculum vit\ae{}}          % use an optional argument to use a string other than "Enclosure", or redefine \enclname
\makelettertitle

Lorem ipsum dolor sit amet, consectetur adipiscing elit. Duis ullamcorper neque sit amet lectus facilisis sed luctus nisl iaculis. Vivamus at neque arcu, sed tempor quam. Curabitur pharetra tincidunt tincidunt. Morbi volutpat feugiat mauris, quis tempor neque vehicula volutpat. Duis tristique justo vel massa fermentum accumsan. Mauris ante elit, feugiat vestibulum tempor eget, eleifend ac ipsum. Donec scelerisque lobortis ipsum eu vestibulum. Pellentesque vel massa at felis accumsan rhoncus.

Suspendisse commodo, massa eu congue tincidunt, elit mauris pellentesque orci, cursus tempor odio nisl euismod augue. Aliquam adipiscing nibh ut odio sodales et pulvinar tortor laoreet. Mauris a accumsan ligula. Class aptent taciti sociosqu ad litora torquent per conubia nostra, per inceptos himenaeos. Suspendisse vulputate sem vehicula ipsum varius nec tempus dui dapibus. Phasellus et est urna, ut auctor erat. Sed tincidunt odio id odio aliquam mattis. Donec sapien nulla, feugiat eget adipiscing sit amet, lacinia ut dolor. Phasellus tincidunt, leo a fringilla consectetur, felis diam aliquam urna, vitae aliquet lectus orci nec velit. Vivamus dapibus varius blandit.

Duis sit amet magna ante, at sodales diam. Aenean consectetur porta risus et sagittis. Ut interdum, enim varius pellentesque tincidunt, magna libero sodales tortor, ut fermentum nunc metus a ante. Vivamus odio leo, tincidunt eu luctus ut, sollicitudin sit amet metus. Nunc sed orci lectus. Ut sodales magna sed velit volutpat sit amet pulvinar diam venenatis.

Albert Einstein discovered that $e=mc^2$ in 1905.

\[ e=\lim_{n \to \infty} \left(1+\frac{1}{n}\right)^n \]

\makeletterclosing

\end{comment}

\begin{comment} %previous software layout
\begin{itemize}
    % TODO: more concrete description
    \item\texttt{\textbf{MetaFast} (\url{https://github.com/ctlab/metafast}):} \\ 
    METAgenome FAST analysis toolkit for feature extraction and calculating a number of statistics of metagenome sequences. \\
    Role: developer, algorithms designer, teamlead.
    \item\texttt{\textbf{MetaCherchant} (\url{https://github.com/ctlab/metacherchant}):} \\
    tool for analysing genomic environment of a nucleotide sequence within a metagenome. \\
    Role: teamlead, algorithms designer.
    \item\texttt{\textbf{RECAST} (\url{https://github.com/ctlab/RECAST}):} \\
    tool for analysing metagenome time series and distinguish which reads of one metagenome sample are found in other samples. \\
    Role: teamlead, algorithms designer.
    \item\texttt{\textbf{GADMA} (\url{https://github.com/ctlab/GADMA}):} \\
    tool for automatic inference of the joint demographic history of multiple populations from the genetic data. \\
    Role: algorithms part supervisor.
    \item\texttt{\textbf{EFSM-tools} (\url{https://github.com/ulyantsev/EFSM-tools}):} \\
    toolset for finite-state machine (FSM) synthesis.
    Role: developer, algorithms designer, teamlead.
\end{itemize}
\end{comment}


\end{document}
